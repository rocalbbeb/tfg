\chapter{Introducción:}

\section{Contextualización del problema energético en IA}

La inteligencia artificial (IA) ha experimentado un crecimiento exponencial en los últimos años, lo que ha provocado importantes implicaciones medioambientales. Según estimaciones recientes de 2023 reflejadas en \cite{cotta2024consumo}, los centros de datos dedicados al procesamiento de IA representan entre el 5\% y el 9\% de la demanda mundial de electricidad y generan aproximadamente el 2\% de las emisiones globales de CO2. Este impacto ambiental ha convertido la optimización energética en un aspecto crítico para el desarrollo de algoritmos de IA.

A medida que el sector de la inteligencia artificial avanza a un ritmo acelerado, las leyes y regulaciones destinadas a controlar su impacto medioambiental evolucionan de forma más lenta. Como consecuencia, muchos proyectos de IA no aplican criterios de sostenibilidad en su desarrollo. A esta tendencia de creación y uso de tecnologías de IA sin considerar su huella ambiental se la conoce como Red Computing.\cite{zhou2023opportunities}

\section{Importancia de la optimización energética en algoritmos evolutivos}

\section{Preguntas de investigación}

\section{Hipótesis y objetivos}

\section{Estructura de la memoria}